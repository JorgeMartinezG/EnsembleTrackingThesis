% Chapter Template

\chapter{Introduction} % Main chapter title

\label{Chapter1} % Change X to a consecutive number; for referencing this chapter elsewhere, use \ref{ChapterX}

\lhead{Chapter 1. \emph{Introduction}} 

Visual object tracking is an important problem in computer vision. This field has a wide range of applications such as surveillance,en success building trackers for specific object classes, a generic object tracker represents a challenging task. In general case, tracking an arbitrary object in an unknown scenario is still considered unsolved. Common challenges are for example, object deformations, illumination human-computer interaction and motion analysis. Althought there has be changes, partial and complete occlusions, drifting, background clutter and similar objects in the scene. This particular situations make some trackers better than others. However, there is no algorithm that has mastered all possible problems that a scenario might generate.

Recently, the evaluation performed in \cite{Wu2013} shows, each tracking algorithm performs well on different sequences. This explains that different tracking algorithms avoid challenges that can occur in general object tracking. We consider an approach that combines the virtues of different algorithms while evading their weaknesses could outperform each single algorithm. Just as "two heads are better than one", making trackers perform this task together in an unknown scenario, may result in a higher level of performance and achievement, than could be obtained individually. This is what in psychology states as "positive interdependence", the ability of group members to encourage and facilitate each other's efforts \cite{Johnson1998}. 


\section{Goals}

In this paper, we focus on the problem of tracking an arbitrary object in videos, with no prior knowledge other than its location in the first frame, also known as "model free tracking". We are motivated to link trackers together so one cannot succeed, unless all group members succeed. A common tracking system consists of an appearance model, which can evaluate the likelihood of the object of interest at a given location. A motion model, that stores and contains the locations of the object over time. Finally, a search strategy to find the best location in the current frame \cite{Yilmaz2006}.All methods share the same goal, meaning that each tracker's individual "effort" is required and is indispensable for group success. Using these models, we make trackers correct each other, increasing performance and ensuring the group is united to a common goal, a concrete reason of being, a purpose for existence.
\chapter{Conclusion and Future Work}
\label{chapter::conclusions}

In this thesis, we demonstrated that an online ensemble of trackers can result
in improved performance with respect to each individual tracker. Our framework
considers the spatial coherence and appearance of the predicted target locations
to ensemble a final estimate of the target state. We leverage high confidence
estimation to reinitialize trackers that fail and steer them towards the true
region. Our experiments show that our simple but effective technique can achieve
state-of-the-art tracking performance in benchmarking sequences. 

Using a quite simple framework, further improvements are expected to be
possible. For instance, the ensemble method does not estimate scale. This
limitation needs to be addressed in future research, since it might give
significant performance gain. One possible solution could be the application of
a sliding window in the scale dimension, preserving low computational cost.

Other possible futures research are proposed. We plan to enlarge evaluation of
the ensemble of trackers using large scale evaluation benchmark. Moreover, we
expect to formulate a novel appearance model update, that will perform this task more efficiently, avoiding spurious samples aggregation into the model. Finally,
we propose to apply better object modeling using more robust features.